\chapter{Conclusion}
\label{chapter:conclusion}

Our main goal of this thesis to specify a runtime configuration registry for \gls{gl:riot_os} has been achieved (see \autoref{chapter:design}).
The implementation (see \autoref{chapter:implementation}) has also been successful and fulfills all the requirements specified in \autoref{chapter:requirements}.

Looking back and around to see what has already been done prior to this thesis in \autoref{chapter:related_work} has been very helpful for us to get an overview of in what ways \glspl{ac:rcs} have been designed so far and for which use-cases.
Thinking about the use-cases of already existing solutions helped in specifying use-cases and requirements that are relevant to fulfill, for the \gls{gl:riot_os} implementation even if the related work was not exactly what is needed for a \gls{gl:riot_os} \gls{ac:rcs}.
This in turn helped us to assess how much the in \autoref{chapter:related_work} assessed already existing implementations are capable of fulfilling these \gls{gl:riot_os}-specific requirements (see \autoref{chapter:requirements} and \autoref{chapter:related_work} \autoref{sec:related_work:assessment_of_implementation_work_on_thesis_requirements}).
We learned that none of the assessed already existing implementations fulfill all the requirements of \autoref{chapter:requirements}, but we were able to base the new \nameref{sec:design:riot_registry} on the existing \gls{ac:mynewt_config} implementation for \gls{gl:riot_os}, which helped speed up the development process by a lot.
Our final design (see \autoref{chapter:design}) turned out much different to from where it started (\gls{ac:mynewt_config}), but the main \gls{ac:api} functions are still roughly the same.

Writing unit tests (see \autoref{chapter:testing}) helped not only finding a lot of bugs in the \nameref{sec:design:riot_registry} implementation, but also helped in creating new ones while further iterating on the \nameref{sec:design:riot_registry}'s implementation.
Which is why, it is of high importance to further increase the testing coverage as mentioned in \autoref{chapter:future_work}.

\autoref{chapter:evaluation} showed us that our implementation's overhead is acceptable for devices with enough memory (stack overhead up to > 4 kilobyte) and it does not perform any dynamic heap allocations, but there are still many ways to further reduce its stack overhead.

For the future it will be very interesting to see how good the new \nameref{sec:design:riot_registry} can specify \glspl{ac:cs} that abstract common configuration parameters shared by the many \gls{gl:riot_os} modules and drivers (see \autoref{sec:future_work:specification_of_schemas_for_common_use_cases}).

\nocite{never_gonna_give_you_up}
