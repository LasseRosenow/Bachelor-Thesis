% !TEX root = ../thesis.tex
%
% configurations
%

% English Language support
% -> uncomment if needed
% Beta!
\fullenglish{yes}
%\fullenglish{no}

% text field
%-> replace supervisor names with correct ones
\firstSupervisor{Prof. Dr. Thomas Schmidt}
\secondSupervisor{Prof. Dr. Franz Korf}

% text field
%-> replace title with your thesis title
\thesisTitle{Laufzeitkonfiguration von eingeschränkten Geräten über ein gemeinsames Betriebssystemmodul}
\thesisTitleEN{Runtime configuration of constrained devices via a shared operating system module}

% text field
%-> replace the key words with your own key words
\keywordsDE{Laufzeitkonfiguration, Eingebettete Geräte, IoT, Betriebssystem}
\keywordsEN{Runtime Configuration, Constrained Devices, IoT, Operating System}

% text field
%-> replace the text with a description of the thesis
\abstractDE{
    Viele Anwendungen im Internet der Dinge (IoT) verwenden Parameter, die zur Laufzeit geändert werden müssen.
    Typische Beispiele können Authentifizierungsdaten, die Abtastrate einer Messung oder eine LED-Farbe sein.
    Darüber hinaus sind in modernen Industrieumgebungen Laufzeitparameter von Geräten über Plug-and-Play-fähige Protokolle wie \gls{gl:lwm2m} usw. fernveränderbar.

    Gegenwärtig stellt \gls{gl:riot_os} keine \acrshort{ac:api} für Laufzeitparameter bereit.
    Stattdessen muss jede Anwendung benutzerdefinierte, oft redundante Logik für ihren spezifischen Anwendungsfall implementieren.
    Eine voll funktionsfähige Integration externer Konfigurationsmanager wie \gls{gl:lwm2m} erweist sich angesichts der enormen Menge an Treibern und Geräten, die für jedes einzelne Verwaltungstool unterstützt und gewartet werden müssen, als besonders schwierig.

    In dieser Arbeit wird eine Konfigurationsregistrierung auf Systemebene eingeführt, mit der man Parameterwerte zur Laufzeit setzen und abzurufen kann.
    Sie organisiert die Parameter nach einem standardisierten Schema, das Typinformationen und andere Metadaten wie Name und Beschreibungsfelder enthält, um sie anderen Modulen wie Konfigurationsmanagern zugänglich zu machen.
    Die Parameterwerte werden optional im lokalen Speicher gehalten.
    Um die Parameter zu ändern, werden eine CLI und Schnittstellen zu \acrshort{ac:coap}, \gls{gl:mqtt} und \gls{gl:lwm2m} spezifiziert.
    Darüber hinaus werden in dieser Arbeit wichtige Entscheidungen zum \acrshort{ac:api}-Design diskutiert und erläutert, warum standardisierte, modulunabhängige Schemata für gemeinsame Konfigurationsparameter für die Integration von externen Management-Tools wie \gls{gl:lwm2m} notwendig sind.
}
\abstractEN{
    Many applications on the Internet of Things (IoT) use parameters that need to be changed at runtime.
    Typical examples can be authentication credentials, the sampling rate of a measurement or an LED color.
    Furthermore in modern industrial spaces runtime parameters of devices are remotely changeable using plug-and-play capable protocols such as \gls{gl:lwm2m} etc.

    As of today, \gls{gl:riot_os} does not provide an \acrshort{ac:api} for runtime parameters.
    Instead, each application needs to implement custom, often redundant logic for its specific use case.
    A fully featured integration of external \glspl{gl:configuration_manager} such as \gls{gl:lwm2m} proves to be particularly difficult given the vast amount of drivers and devices that need to be supported and maintained for every single management tool.

    In this thesis we introduce a system-level configuration registry that allows for setting and getting configuration parameter values at runtime.
    It organizes parameters according to a standardized schema that includes type information and other metadata such as name and description fields to make them accessible to other modules such as \glspl{gl:configuration_manager}.
    The parameter values are optionally persisted on local storage.
    To change the parameters we specify a CLI and interfaces to \acrshort{ac:coap}, \gls{gl:mqtt} and \gls{gl:lwm2m}.
    Furthermore, we discuss important \acrshort{ac:api} design decisions and explain why standardized, module-independent schemas for common configuration parameters are essential for the integration of external management tools such as \gls{gl:lwm2m}.
}

% text field
%-> replace john with your name
\thesisAuthor{Lasse Jonas Rosenow}

% text field
%-> enter the submission date
\submissionDate{February 1, 2023}

% switch - uncomment only one
%-> uncomment NDA or public
%\NDA{yes}
\NDA{no}

% switch - uncomment only one
%-> uncomment old standard cover or cover Corporate Design 2017
\Cover{CD2017}
%\Cover{CD2017NoLogo}
%\Cover{Std2018}
%\Cover{Std2018_green} 			% with green bar

% switch - uncomment only one
%-> uncomment to show list of figures or not
\ListOfFigures{yes}
%\ListOfFigures{no}

% switch - uncomment only one
%-> uncomment to show list of tables or not
\ListOfTables{yes}
%\ListOfTables{no}

% switch - uncomment only one
%-> uncomment to show list of listings or not
\ListOfListings{yes}
%\ListOfListings{no}

% switch - uncomment only one
%-> uncomment to show list of acronyms or not
\ListOfAccronyms{yes}
%\ListOfAccronyms{no}

% switch - uncomment only one
%-> uncomment to show list of symbols or not
\ListOfSymbols{yes}
%\ListOfSymbols{no}

% switch - uncomment only one
%-> uncomment to show list of glossary entries or not
\Glossary{yes}
%\Glossary{no}

% switch - uncomment only one
%-> uncomment the study course you are in
\studycourse{ITS}
%\studycourse{TI}
%\studycourse{AI}
%\studycourse{WI}
%\studycourse{EI}
%\studycourse{REE}
%\studycourse{BMP}		
%\studycourse{BMP-hp}	 % Internship Report in M&P
%\studycourse{BMT}
%\studycourse{BMT-st}    % Study / home assignment in BMT
%\studycourse{BMT-hp}    % Internship Report in BMT
%\studycourse{MI}
%\studycourse{MIK}
%\studycourse{MA}
